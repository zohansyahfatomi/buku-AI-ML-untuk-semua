\preface
Segala puji dan syukur penulis sampaikan ke hadirat Alloh SWT, Tuhan semesta alam yang telah melimpahkan segala kenikmatan dan kesempatan sehingga penulis dapat menyelesaikan penyusunan buku Pemrograman dan Komputasi Numerik : Menggunakan Python.  Buku ini menyajikan pembahasan terkait dengan dasar pemrograman dan penerapannya pada komputasi numerik. Buku ini ditulis sebagai panduan bagi pembaca secara umum khususnya bagi pembaca dengan background sains dan teknik. Bagi pembaca yang belum mengerti terkait bahasa pemrograman bisa memulai belajar pemrograman dari awal melalui buku ini. Bagi pembaca yang mempunyai pengetahuan pemrograman, buku ini dapat digunakan untuk memperdalam kemampuan pemrograman dan juga sekaligus untuk memahami terkait komputasi numerik.  Dalam buku berikut ini, penulis menggunakan Python karena selain cukup powerfull untuk komputasi numerik, Python merupakan software pemrograman mudah untuk dipelajari ditunjukkan dengan deklarasi program (code) yang tidak panjang. Selain karena motivasi kemudahan dalam pembelajaran, penulis menggunakan Python karena software pemrograman ini sedang paling banyak (ingin) dikuasai oleh pengembang pemrograman di seluruh dunia. Harapan penulis, selain pembaca mendapatkan ilmu terkait komputasi numerik, dengan pemrograman Python pembaca dapat mengembangkan ilmunya dalam bidang lain seperti Data Science atau pengembangan aplikasi baik berbentuk mobile-based, 
dekstop-based maupun web-based. Sebagai suatu karya, buku ini tentunya belum sempurna, sehingga masukan dari berbagai pihak sangat diperlukan. Terakhir, semoga buku ini bermanfaat bagi bagi pembaca dan berkontribusi dalam penyebaran khazanah keilmuan dan pendidikan. 

\begin{tabular}{p{7.5cm}c}
%	tulis pra kata disini
&Yogyakarta, 20 Maret 2020\\
\\
&\\
&Penulis
\end{tabular}