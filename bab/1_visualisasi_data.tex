%----------------------------------------------------------------------------------------
%	CHAPTER 2: Dasar Pemrograman
%----------------------------------------------------------------------------------------
%\chapterimage{head1.png}

\chapter{VISUALISASI DATA}
\section{Apa itu Data?}
Data adalah representasi nyata dari suatu objek. Data dapat berbentuk teks, numerik, gambar (warna), suara, dlsb.

\section{Fungsi Data}
\begin{enumerate}
	\item Sebagai acuan untuk melakukan pengkajian.
	\item Sumber kajian dan evaluasi dari suatu aktivitas.
	\item Acuan dalam menentukan kebijakan/keputusan untuk kedepannya.
\end{enumerate}

\section{Jenis Data}
Pada dasarnya data dapat dibedakan menjadi 2 :
\begin{enumerate}
	\item Data Kualitatif
	\item Data Kuantitatif
\end{enumerate}

\subsection{Data Kualitatif}
Data kualitatif merupakan representasi obyek yang berbentuk pernyataan kwalitas. Biasanya data kualitatif diformasi dari rekam pendapat yang sifatnya \textbf{subyektif}. Artinya, bisa saja antara 2 orang yang berbeda dapat memberikan pemberian data yang berbeda dalam 1 obyek yang sama.

Semisal terdefinisi seekor sapi. Sapi tersebut memilki data kualitatif sebagai berikut.
\begin{table}[hbt!]
  \centering
   \caption{Data kualitatif pada sapi}
   \label{tab:1-1}
  \begin{tabular}{ p{3cm} p{7cm} }
    \hline\hline
    Data & Deskripsi\\
    \hline
    Keadaan & Sehat \\
    Warna & Hitam-putih \\
    Tekstur & Lembut \\
    \hline 
  \end{tabular}
\end{table}

Adapun metode pengumpulan data adalah sebagai berikut.
\begin{enumerate}
	\item Pertanyaan (wawancara)
	\item Diskusi
	\item Observasi
\end{enumerate}

Analisa data kualitatif.
Tidak ada aturan yang pasti terkait dengan metode analisa pada data kualitatif. Mengapa? Hal tersebut dikarenakan data dianggap sesuatu yang telah benar (subyektif). Adapun metode yang sering kali digunakan adalah sebagai berikut :
\begin{enumerate}
	\item Pendekatan Deduktif 
	Strukur penelitian ditentukan oleh peniliti. Cepat dan Mudah.
	\item Pendekatan Induktif
	Pendekatan ini tidak secara langsung ditentukan oleh peneliti. Lama dan susah.
\end{enumerate}

\subsection{Data Kuantitatif}
Data kuantitatif adalah data yang bisa direpresentasikan dalam bentuk angka-angka (nilai numerik). Data kuantitatif dapat dihasilkan dari suatu eksperimen (pengukuran maupun observasi). Data kuantitatif relatif lebih bersifat obyektif karena didapatkan dari hasil pengukuran bukan dari pendapat seseroang. 

Semisal terdefinisi seekor sapi. Sapi tersebut memilki data kualitatif sebagai berikut.
\begin{table}[hbt!]
  \centering
   \caption{Data kuantiatif pada sapi}
   \label{tab:1-1}
  \begin{tabular}{ p{3cm} p{7cm} }
    \hline\hline
    Data & Deskripsi\\
    \hline
    Berat & 300 kg \\
    Panjang & 3 meter \\
    Harga & 16.000.000 \\
    \hline 
  \end{tabular}
\end{table}

Adapun metode pengumpulan data adalah sebagai berikut.
\begin{enumerate}
	\item Pertanyaan (wawancara)
	\item Sampling probabilitas
	\item Survei atau kuesioner
	\item Pengamatan
\end{enumerate}

Analisa data kuantitatif :
\begin{enumerate}
	\item Analisa Maxdiff 
	\item Tabulasi Silang
	\item Analisa SWOT
\end{enumerate}

Langkah analisa data kuantitatif:
\begin{enumerate}
	\item Hubungkan skala pengukuran dengan variabel
	\item Hubungkan statistif dekskriptif dengan data
	\item Tentukan skala pengukuran 
	\item Pemilihan tabel 
\end{enumerate}

\section{Data Diskrit vs Data Kontinyu}
Dari perilaku datanya data dapat dibedakan menjadi 2 :
\begin{enumerate}
	\item Data diskrit
	\item Data kontinyu
\end{enumerate}

\subsection{Data Diskrit}
Data diskrit merupakan data yang terkumpul hasil dari pencacahan. Contoh data jumlah kehadiran  orang dalam suatu acara atau jumlah peliharaan ayam Anda di kandang. Pada data diskrit Anda tidak mungkin mendapatkan nilai desimal, Anda tidak mungkin menghitung kehadiran pada suatu kelompok dengan jumlah 11.5 orang.

Adapun karakteristik dari data diskrit adalah sebagai berikut :
\begin{enumerate}
	\item Dapat dihitung
	\item Nilai tidak mungkin menjadi desimal (pecahan)
	\item Tidak dapat diukur (menggunakan alat)
	\item Biasanya direpresentasikan dalam bentuk grafik batang
\end{enumerate}

\subsection{Data Kontinyu}
Data kontinyu merupakan data yang terkumpul dari hasil perhitungan. Semisal Anda ingin mengetahui panjang dari suatu meja. Anda akan mengukurnya menggunakan mistar bukan mencacahnya bukan? Hasil yang didapatkan pun bisa berbentuk desimal (pecahan), mungkin panjang dari meja Anda dapat bernilai 100.5 cm,

Adapun karakteristik dari data kontinyu adalah sebagai berikut :
\begin{enumerate}
	\item Dapat diukur (menggunakan alat)
	\item Nilai mungkin menjadi desimal (pecahan)
	\item Tidak dapat dicacah
	\item Biasanya direpresentasikan dalam bentuk grafik histogram
\end{enumerate}

\section{Data Numerik vs Data Kategori}
Data juga dapat dibedakan berdasarkan nilai variabelnya, yaitu:
\begin{enumerate}
	\item Data Numerik
	\item Data Kategori
\end{enumerate}

\subsection{Data Numerik}
Data yang identik dengan nilai numerik. Data numerik dapat dibedakan menjadi data diskrit dan kontinyu.

\subsection{Data Kategori}
Data yang bisa dikategorikan berdasarkan karaketeristik masing-masing individu data. Contohnya status perkawinan data di KTP, agama, kota asal, hobi, jenis kelamin dlsb.
\subsubsection{Data Nominal}
Data nominal merupakan data yang tidak memiliki korelasi atau kerterkaitan dengan data lainnya. Contoh : jenis kelamin, pekerjaan dlsb.

Ciri-ciri data nominal :
\begin{enumerate}
	\item Hasil pencacahan (tidak dalam bentuk desimal)
	\item Angka yang mucnul hanya digunakan sebagai penera (simbol)
	\item Simbol dapat berbentuk angka, huruf, lambang dlsb.
\end{enumerate}

\subsubsection{Data Ordinal}
Data ordinal merupakan suatu bilangan diskrit yang memiliki derajat urutan tertentu. Contohnya waktu yang muncul sebagai rangking orang berlari. Orang A : 1 jam, Orang B : 1 jam 10 menit, orang C : 1 jam 11 menit dlsb.

\section{Data Terstrukur vs Data Tidak Terstruktur}
\subsection{Data Terstruktur}
Biasanya terformat dalam tabel. Contohnya data excel, data RDBMS, kartu stok dlsb.
\subsection{Data Tidak Terstruktur}
Lebih beragam bentuknya. Contohnya citra image pada satelit, noise dari sadap, dlsb.
\subsection{Data Semi Terstruktur}
XML, file .csv, JSON dlsb.

\section{Visualisasi Data}
Cara mengkomunasikan data dengan format visual tertentu, semisal diagram, grafik atau representasi lainnya. 

Tujuan visualisasi data :
\begin{enumeraate}
	\item Komunakasi lebih efektif.
	\item Pemantauan data lebih mudah.
\end{enumerate}

Media visualisasi data :
\begin{enumeraate}
	\item Tabel. 
	Perhatikan : Judul, kesederhaan, penjelasan simbol, penekanan pada suatu data tertentu, sumber tabel.
	\item Diagram.
	Jenis : diagram batang, diagram garis, diagram lingkaran.
\end{enumerate}

%copas
Visualisasi Data dalam bisnis :
\begin{enumeraate}
	\item Dashboard.
	Dashboard merupakan kumpulan dari berbagai visualisasi yang berbeda yang menggabungkan dan merangkum informasi atau data bisnis. 
	\item Scoreboard
	Scorecard merupakan tipe lainnya dalam visualisasi di bidang bisnis. Berbeda dengan dashboard yang terdapat banyak visualisasi di dalamnya, scorecard lebih fokus pada sebuah target tertentu. Visualisasinya berupa  jumlah pendapatan, kepuasan pelanggan, dan hal lainnya yang dapat dibandingkan dengan target yang telah ditentukan. Scorecard juga biasanya disajikan dalam salah satu komponen dashboard. Scorecard menggambarkan tentang Key Performance Indicators (KPI) yang lebih disederhanakan untuk dapat memantau kemajuan progres.
	\item Analytic report
	Analytic report atau laporan yang berisi analisis yang digunakan untuk menentukan keputusan. Jenis laporan ini menggunakan data kualitatif dan kuantitatif untuk menganalisis dan mengevaluasi ide dari suatu bisnis. Analytic report memberikan keuntungan untuk pembaca karena memberi  pemahaman yang mudah dipahami. Selain itu hanya dengan membaca sekilas saja. pembaca juga dapat memahami data dalam jumlah yang banyak. 

	Analytic report juga menerapkan langkah-langkah umum seperti mengidentifikasi masalah, menentukan metode yang tepat, analisis data, dan mendapatkan solusi terbaik dari masalah yang dihadapi.
	\item Report
	Report merupakan bagian dari visualisasi data dalam bisnis yang memuat semua ringkasan dari apa yang terjadi di perusahaan dalam waktu tertentu. Inti dari report adalah apa yang Anda lakukan untuk mendapatkan dan memahami hal yang sedang terjadi dari suatu perusahaan dengan secepat mungkin.
\end{enumerate}

Tool yang bisa Anda gunakan :
\begin{enumeraate}
	\item Tableau Public
	Tableau Public merupakan sebuah layanan gratis yang memungkinkan siapa saja dapat mempublikasikan visualisasi data ke dalam web. Visualisasi yang telah dipublikasikan ke Tableau Public ("vizzes") dapat diletakkan dalam halaman web dan blog, dibagikan ke sosial media, dan juga dapat juga diunduh oleh pengguna lainnya. Untuk proses pembuatan visualisasi datanya sendiri menggunakan aplikasi terpisah bernama Tableau Desktop Public Edition. Ingin tahu hal yang menarik dari Tableau Desktop Public ini? Ya, Anda dapat menggunakan aplikasi ini tanpa memerlukan keahlian dalam bidang pemrograman. Keren, bukan? Untuk mengunduh aplikasi Tableau Public dapat klik di sini, ya.
	\item Google Sheets
	Sudah pernah menggunakan Google Sheets sebelumnya? Anda tidak harus melakukan instalasi aplikasi spreadsheet di laptop Anda, karena dalam Google Spreadsheet semuanya tersedia online. Google Sheets menawarkan kumpulan fitur dan fungsi standar spreadsheet application seperti yang ada di Microsoft Excel. Tentunya pada Google Sheets dapat membuat visualisasi sederhana dari data yang kita buat baik dalam bentuk diagram batang, diagram garis, maupun diagram lingkaran.
	\item Microsoft Excel
	Pasti Anda familiar dengan Microsoft Excel, bukan? Sebuah aplikasi spreadsheet buatan Microsoft yang memuat banyak fitur powerful. Microsoft Excel menggunakan spreadsheet yang terdiri dari baris dan kolom untuk manajemen data serta melakukan perhitungan fungsi yang lebih akrab disebut formula. Selain melakukan perhitungan angka yang bersifat numerik, Excel juga dapat membuat visualisasi data sederhana ke dalam bentuk grafik seperti diagram garis, batang, lingkaran, dan lain-lain.
\end{enumerate}
