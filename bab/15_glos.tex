\chapter{GLOSARIUM}
\textbf{Algoritma} Tata cara untuk menyelesaikan masalah secara sistematis.\\ \\
\textbf{Assembly}  Bahasa pemrograman tingkat rendah (satu tingkat di atas bahasa mesin) yang sangat bergantung pada arsitektur mesin tertentu. \\ \\
\textbf{Array}  Variabel yang dapat menampung nilai lebih dari satu namun harus memiliiki tipe data yang sama. \\ \\
\textbf{Compiler} Suatu program khusus yang digunakan untuk mengeksekusi instruksi pada suatu skrip program dengan melakukan kompilasi (penerjemahan pada bahasa tertentu, contohnya bahasa mesin). \\ \\
\textbf{Dict} Salah satu struktur data pada Python yang merupakan variabel multinilai yang komponen nilainya berupa kunci dan nilai (seperti kamus). \\ \\
\textbf{Fungsi} Kumpulan instruksi program yang memiliki sebuah tujuan tertentu. \\ \\
\textbf{IDE} Kepanjangan dari Integrated Development Environtment yaitu gabungan (kesatuan) dari pada interpreter/compiler, library, text-editor, debugging, dlsb yang dikonstruksi untuk memudahkan pengembangan suatu program.\\ \\
\textbf{Interpreter} Suatu program khusus yang digunakan untuk mengeksekusi instruksi pada suatu skrip program tanpa melakukan kompilasi.\\ \\
\textbf{Indentasi} Blok pada program Python yang dicirikan dengan tulisan yang lebih menjorok (hasil eksekusi tombo "tab"). \\
\textbf{Kelas} Cetak biru dari objek.\\ \\
\textbf{Konstanta} Wadah dari suatu nilai namun nilainya tidak berubah.\\ \\
\textbf{Konstruktor} Method khusus yang digunakan untuk memberikan keadaan awal dari suatu objek sejak objek tersebut diinstanisasi (diwujudkan).\\ \\
\textbf{Konvergen} Keadaan dimana hasil iteratif dari suatu perhitungan menuju pada suatu nilai tunggal.\\ \\
\textbf{List} Salah satu struktur data pada Python yang merupakan variabel multinilai yang komponen nilainya bisa berubah.\\ \\
\textbf{Numpy} Salah satu library pada Python yang digunakan untuk komputasi ilmiah berbasis array.\\ \\
\textbf{Matplotlib} Salah satu library pada Python yang digunakan untuk visualisasi data.\\ \\
\textbf{Matriks} Sebuah entitas matematika dimana datanya diatur sedemikian rupa menjadi deret dan kolom.\\ \\
\textbf{Objek} Representasi atau pemodelan "sesuatu" yang eksis di dunia nyata untuk mewadahi suatu atribut (data) dan perilaku (method) dari suatu sistem tertentu.\\ \\
\textbf{Operator} Simbolisme dari suatu perintah manusia kepada komputer (melalui intepreter atau compiler) untuk melakukan suatu operasi khusus seperti operasi aritmatika, operasi logika, operasi relasi, dlsb.\\ \\
\textbf{Jupyter Notebook} Salah satu IDE untuk bahasa Python berbasis web.\\ \\
\textbf{Polimorpisme} Fitur dari paradigma objek yang mengizinkan Anda untuk mengimplementasikan varibel dan method yang sama pada anak kelas dengan cara yang berbeda.\\ \\
\textbf{Program} Kumpulan instruksi untuk melaksanakan suatu tugas tertentu.\\ \\
\textbf{Programer} Seseorang yang memiliki kapabilitas untuk menkonstruksi program untuk menyelesaikan suatu masalah tertentu.\\ \\
\textbf{Pycharm} Salah satu IDE untuk bahasa Python berbasis dekstop.\\ \\
\textbf{Python} Salah satu jenis pemrograman tingkat tinggi yang menggunakan interpreter dengan fitur dynamic-binding dan dynamic-typing.\\ \\
\textbf{Tipe data} Klasifikasi data berdasarkan perilakunya (ukuran data, pengoperasiannya, dlsb)\\ \\
\textbf{Rekursif} Jenis fungsi yang digunakan untuk melaksanakan prinsip pengulangan dengan jalan memanggil dirinya sendiri.\\ \\
\textbf{RPEL} Kepanjangan dari Read Evaluate Print Loop yaitu interaktif interpreter (biasanya berupa tampilan Command Line Interface).\\ \\
\textbf{Sistem Linear} Sistem persamaan yang suku variabel independennya terbatas memiliki derajat pangkat satu.\\ \\
\textbf{Sistem Nonlinear} Sistem persamaan yang suku variabel independennya memiliki derajat pangkat lebih dari satu.\\ \\
\textbf{Set} Salah satu struktur data pada Python yang merupakan variabel multinilai yang komponen nilainya bersifat unik.\\ \\
\textbf{Sistem Operasi} Program (perangkat lunak) yang bertanggung jawab melakukan manajemen kinerja antara perangkat keras dengan perangkat lunak lainnya.\\ \\
\textbf{Tuple} Salah satu struktur data pada Python yang merupakan variabel multinilai yang komponen nilainya tidak bisa berubah.\\ \\
\textbf{Variabel} Wadah dari suatu nilai yang berada di dalam memori RAM.\\ \\

